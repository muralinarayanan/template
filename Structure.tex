\usepackage{graphicx}   
\usepackage{booktabs}
\usepackage{fancyhdr}
\usepackage{float}
\usepackage{wrapfig}
\usepackage{multicol}  % Support for Multiple Columns
\usepackage{multirow}
\usepackage{listings}  % make code looks nicer
\usepackage{amsmath} % for \hookrightarrow
%\usepackage{xcolor}   % for \textcolor
\usepackage{microtype}
\usepackage{colordvi}
\usepackage{rotating} % used to rotate the pictures   % not used anymore

\usepackage{longtable}
\usepackage{tikz}
\usepackage{pdfpages}  % not used any more
\usepackage{pdflscape} 
\usepackage{array}  
\usepackage{listings}
\lstset{
	basicstyle=\small\ttfamily,
	columns=flexible,
	breaklines=true,
	postbreak=\mbox{\textcolor{red} {$\hookrightarrow$}\space}
}


\usepackage{wasysym} % some symbols Like Telephone
\usepackage{amssymb} % need for the Box command
\usepackage{versions}
\usepackage[letterpaper]{geometry}
\geometry{verbose,
	tmargin=1.5in,bmargin=1in,lmargin=1in,rmargin=1in, marginparwidth=0.5in,
	headheight=1in,headsep=0.5in,footskip=0.5in}
%\geometry{verbose,tmargin=1.5in,bmargin=1.5in,lmargin=2in,rmargin=1in,headheight=1in,headsep=0.5in,footskip=1in}

%% Migrate to Arial font for cleaner look.
\renewcommand{\rmdefault}{phv} % Arial
\renewcommand{\sfdefault}{phv} % Arial

\renewcommand{\numberline}[1]{#1~}  % The chapter number does not jumble up in the toc.

%% Even more interesting looking B612 font.
%% Get it from https://b612-font.com/
\usepackage{fontspec}
\setmainfont{B612}
\setsansfont{B612}
\setmonofont{B612 Mono}


\usepackage[unicode=true,
bookmarks=true,bookmarksnumbered=false,bookmarksopen=false,  linktoc=all,
breaklinks=false,pdfborder={0 0 0},pdfborderstyle={},backref=page,colorlinks=false]
{hyperref}

\usepackage{attachfile2}
\attachfilesetup{icon=Paperclip}

% custom chapter begin
\usepackage{titlesec}

\usepackage{makeidx} % for indexing

\usepackage{tocloft} % used for custom toc.

\usepackage{tikz}
\usetikzlibrary{shapes.geometric}
\usetikzlibrary{calc}

\usepackage{anyfontsize}

\usepackage{longtable}
\usepackage{datatool}
\usepackage{booktabs}

\usepackage{progressbar}
% for fun
\usepackage{tikz}
\usetikzlibrary{decorations.text}
% end for fun

\usepackage{anyfontsize}

\usepackage{fontawesome5}


\renewcommand{\numberline}[1]{#1~}  % The chapter number does not jumble up in the toc.



\titleformat{\chapter}[block]
{\normalfont\huge\bfseries}{\thechapter}{20pt}{}
\titlespacing*{\chapter}
{0pt}{20pt}{20pt}
% custom chapter end


%%%  LET us have a nice box for todo and info
\usepackage[most]{tcolorbox} % for use in Colored box

\newcommand{\mytodo}[1] { 
	\begin{tcolorbox}[colback=yellow!5!white,colframe=red!75!black]
		\index{todo} #1
	\end{tcolorbox}	
}

\newcommand{\myinfo}[1] { 
	%	\begin{tcolorbox}[colback=yellow!2!white,colframe=yellow!90!black]
	\begin{tcolorbox}[colback=yellow,colframe=black]
		#1
	\end{tcolorbox}	
}


\newcommand{\myblackbox}[1] { 
	\begin{tcolorbox}[colback=black,colframe=white,coltext=white]
		#1
	\end{tcolorbox}	
}



\usepackage[framemethod=tikz]{mdframed} % Allows defining custom boxed/framed environments
\mdfdefinestyle{info}{%
	topline=false, bottomline=false,
	leftline=false, rightline=false,
	nobreak,
	singleextra={%
		\fill[black](P-|O)circle[radius=0.4em];
		\node at(P-|O){\color{white}\scriptsize\bf i};
		\draw[very thick](P-|O)++(0,-0.8em)--(O);%--(O-|P);
	}
}
% Define a custom environment for information
\newenvironment{info}[1][Info:]{ % Set the default title to "Info:"
	\medskip
	\begin{mdframed}[style=info]
		\noindent{\textbf{#1}}
	}{
	\end{mdframed}
}

\newcommand{\Minutes}[2][\today]{
	\begin{tcolorbox}[colback=gray!15!white,colframe=black,coltext=black, title=#1]
		#2
	\end{tcolorbox}	
	
}


%----------------------------------------------------------------------------------------
%	FILE CONTENTS ENVIRONMENT
%----------------------------------------------------------------------------------------

% Usage:
% \begin{file}[optional filename, defaults to "File"]
%	File contents, for example, with a listings environment
% \end{file}

\mdfdefinestyle{file}{
	innertopmargin=1.6\baselineskip,
	innerbottommargin=0.8\baselineskip,
	topline=false, bottomline=false,
	leftline=false, rightline=false,
	leftmargin=1cm,
	rightmargin=1cm,
	singleextra={%
		\draw[fill=black!10!white](P)++(0,-1.2em)rectangle(P-|O);
		\node[anchor=north west]
		at(P-|O){\ttfamily\mdfilename};
		%
		\def\l{3em}
		\draw(O-|P)++(-\l,0)--++(\l,\l)--(P)--(P-|O)--(O)--cycle;
		\draw(O-|P)++(-\l,0)--++(0,\l)--++(\l,0);
	},
	nobreak,
}

% Define a custom environment for file contents
\newenvironment{file}[1][File]{ % Set the default filename to "File"
	\medskip
	\newcommand{\mdfilename}{#1}
	\begin{mdframed}[style=file]
	}{
	\end{mdframed}
	\medskip
}






\widowpenalty=300
\clubpenalty=300


%%% Compact Enumerations/bullets
\usepackage{enumitem}
\setitemize{noitemsep,topsep=0pt,parsep=0pt,partopsep=0pt}
\setlist[description]{noitemsep,topsep=0pt,parsep=0pt,partopsep=0pt}
\setenumerate{noitemsep,topsep=0pt,parsep=0pt,partopsep=0pt}




%\usepackage{pagecolor}
%\pagecolor{black}
%\color{green!50!white}

\definecolor{MyWhite}{RGB}{255,255,255}
\definecolor{MyBlue} {RGB} {25,49,227}
\definecolor{MyOrange} {RGB} {204,51,51}
\definecolor{MyGreen} {RGB} {33,177,75}

